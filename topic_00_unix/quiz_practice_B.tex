\documentclass[10pt]{article}

\usepackage[margin=1in]{geometry}
\usepackage{amsmath}
\usepackage{amssymb}
\usepackage{amsthm}
\usepackage{mathtools}
\usepackage[shortlabels]{enumitem}
\usepackage[normalem]{ulem}
\usepackage{courier}

\usepackage{hyperref}
\hypersetup{
  colorlinks   = true, %Colours links instead of ugly boxes
  urlcolor     = black, %Colour for external hyperlinks
  linkcolor    = blue, %Colour of internal links
  citecolor    = blue  %Colour of citations
}

\usepackage[T1]{fontenc}
\usepackage{listings}
\lstset{
    language=HTML
    ,basicstyle=\linespread{1}\ttfamily
    %,numbers=left
    ,breaklines=true
    }

%%%%%%%%%%%%%%%%%%%%%%%%%%%%%%%%%%%%%%%%%%%%%%%%%%%%%%%%%%%%%%%%%%%%%%%%%%%%%%%%

\theoremstyle{definition}
\newtheorem{problem}{Problem}
\newtheorem{note}{Note}
\newcommand{\E}{\mathbb E}
\newcommand{\R}{\mathbb R}
\DeclareMathOperator{\Var}{Var}
\DeclareMathOperator*{\argmin}{arg\,min}
\DeclareMathOperator*{\argmax}{arg\,max}

\newcommand{\trans}[1]{{#1}^{T}}
\newcommand{\loss}{\ell}
\newcommand{\w}{\mathbf w}
\newcommand{\mle}[1]{\hat{#1}_{\textit{mle}}}
\newcommand{\map}[1]{\hat{#1}_{\textit{map}}}
\newcommand{\normal}{\mathcal{N}}
\newcommand{\x}{\mathbf x}
\newcommand{\y}{\mathbf y}
\newcommand{\ltwo}[1]{\lVert {#1} \rVert}

%%%%%%%%%%%%%%%%%%%%%%%%%%%%%%%%%%%%%%%%%%%%%%%%%%%%%%%%%%%%%%%%%%%%%%%%%%%%%%%%

\begin{document}
\begin{center}
    {
\Large
    Quiz: Terminal 0 (Practice B)
}

    %\vspace{0.1in}
    %CSCI046: Data Structures

    \vspace{0.1in}
\end{center}

\vspace{0.15in}
\noindent
\textbf{Total Score:} ~~~~~~~~~~~~~~~/$2^2$

\vspace{0.2in}
\noindent
\textbf{Printed Name:}

\noindent
\rule{\textwidth}{0.1pt}
\vspace{0.15in}

\noindent
\textbf{Quiz rules:}
\begin{enumerate}
    \item You MAY use any printed or handwritten notes.
    \item You MAY NOT use a computer or any other electronic device.
\end{enumerate}

\noindent

\vspace{0.15in}


\begin{problem}
    Write the output of the final command in the following terminal session.
    If the command has no output, then leave the problem blank.
\end{problem}
\begin{lstlisting}
$ cd
$ rm -rf quiz
$ mkdir quiz
$ cd quiz
$ git init
$ touch README
$ mkdir foo
$ touch foo/README2
$ touch README3
$ ls -a foo
\end{lstlisting}
\vspace{1in}


\begin{problem}
    Write the output of the final command in the following terminal session.
    If the command has no output, then leave the problem blank.
\end{problem}
\begin{lstlisting}
$ cd
$ rm -rf quiz
$ mkdir quiz
$ cd quiz
$ git init
$ mkdir foo
$ echo "hello world" > foo/README
$ echo "hola mundo" > foo/README_ES
$ echo "hello again" >> foo/README
$ echo "hola otra vez" > README_ES
$ cat foo/README
\end{lstlisting}
\vspace{1in}

\begin{problem}
    Write the output of the final command in the following terminal session.
    If the command has no output, then leave the problem blank.
\end{problem}
\begin{lstlisting}
$ cd
$ rm -rf quiz
$ mkdir quiz
$ cd quiz
$ git init
$ mkdir foo
$ echo "print('hello world')" > foo.py
$ echo "print('hola mundo')" >> foo/bar.py
$ echo 'print("hola otra vez")' >> bar.py
$ echo "print(\"hello again\")" >> foo/foo.py
$ cat foo/foo.py
\end{lstlisting}
\vspace{2in}

\begin{problem}
    Write the output of the final command in the following terminal session.
    If the command has no output, then leave the problem blank.
\end{problem}
\begin{lstlisting}
$ cd
$ rm -rf quiz
$ mkdir quiz
$ cd quiz
$ git init
$ mkdir master
$ echo "print('hello world')" > master/foo.py
$ echo "print('hola mundo')" > foo.py
$ git add master
$ git commit -m "added master"
$ git branch foo
$ git checkout foo
$ echo "print('hola mundo')" >> foo.py
$ git add foo.py
$ git commit -m "modified foo"
$ git checkout master
$ ls
\end{lstlisting}

\end{document}
