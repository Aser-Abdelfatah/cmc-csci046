\documentclass[10pt]{article}
\usepackage[utf8]{inputenc}

\usepackage[margin=1in]{geometry}
\usepackage{amsmath}
\usepackage{amssymb}
\usepackage{amsthm}
\usepackage{mathtools}
\usepackage{bm}
\usepackage{tikz}

\usepackage{color}
\usepackage{colortbl}
\definecolor{deepblue}{rgb}{0,0,0.5}
\definecolor{deepred}{rgb}{0.6,0,0}
\definecolor{deepgreen}{rgb}{0,0.5,0}
\definecolor{gray}{rgb}{0.7,0.7,0.7}

\usepackage{hyperref}
\hypersetup{
  colorlinks   = true, %Colours links instead of ugly boxes
  urlcolor     = black, %Colour for external hyperlinks
  linkcolor    = blue, %Colour of internal links
  citecolor    = blue  %Colour of citations
}

%%%%%%%%%%%%%%%%%%%%%%%%%%%%%%%%%%%%%%%%%%%%%%%%%%%%%%%%%%%%%%%%%%%%%%%%%%%%%%%%

\theoremstyle{definition}
\newtheorem{problem}{Problem}
\newtheorem{example}{Example}
\newtheorem{defn}{Definition}
\newcommand{\E}{\mathbb E}
\newcommand{\R}{\mathbb R}
\DeclareMathOperator{\nnz}{nnz}
\DeclareMathOperator{\determinant}{det}
\DeclareMathOperator{\Var}{Var}
\DeclareMathOperator{\rank}{rank}
\DeclareMathOperator*{\argmin}{arg\,min}
\DeclareMathOperator*{\argmax}{arg\,max}

\newcommand{\trans}[1]{{#1}^{T}}
\newcommand{\loss}{\ell}
\newcommand{\w}{\mathbf w}
\newcommand{\x}{\mathbf x}
\newcommand{\y}{\mathbf y}
\newcommand{\lone}[1]{{\lVert {#1} \rVert}_1}
\newcommand{\ltwo}[1]{{\lVert {#1} \rVert}_2}
\newcommand{\lp}[1]{{\lVert {#1} \rVert}_p}
\newcommand{\linf}[1]{{\lVert {#1} \rVert}_\infty}
\newcommand{\lF}[1]{{\lVert {#1} \rVert}_F}

\usepackage{listings}

% Default fixed font does not support bold face
\DeclareFixedFont{\ttb}{T1}{txtt}{bx}{n}{12} % for bold
\DeclareFixedFont{\ttm}{T1}{txtt}{m}{n}{12}  % for normal

% Python style for highlighting
\newcommand\pythonstyle{\lstset{
language=Python,
basicstyle=\ttm,
otherkeywords={self},             % Add keywords here
keywordstyle=\ttb\color{deepblue},
emph={MyClass,__init__},          % Custom highlighting
emphstyle=\ttb\color{deepred},    % Custom highlighting style
stringstyle=\color{deepgreen},
frame=tb,                         % Any extra options here
showstringspaces=false,           % 
stepnumber=1,
numbers=left
}}

\lstnewenvironment{python}[1][]
{
    \pythonstyle
    \lstset{#1}
}
{}

%%%%%%%%%%%%%%%%%%%%%%%%%%%%%%%%%%%%%%%%%%%%%%%%%%%%%%%%%%%%%%%%%%%%%%%%%%%%%%%%

\begin{document}

\begin{center}
    {
\Large
CSCI046 Notes: Runtime Analysis
}

    %\vspace{0.1in}
%CSCI046, Mike Izbicki
\end{center}


%%%%%%%%%%%%%%%%%%%%%%%%%%%%%%%%%%%%%%%%%%%%%%%%%%%%%%%%%%%%%%%%%%%%%%%%%%%%%%%%

\section{Counting}
Our goal is to count how many times something will happen in our code.
This is often used as a proxy for how long it takes a program to run.

\begin{example}~

\begin{python}
print('x')
print('x')
print('x')

for i in range(10):
    print('y')

for i in range(10,20):
    print('z')
    print('z')
    print('z')
\end{python}
    \begin{enumerate}
        \item What is the exact number of times that the letter \texttt{x} will be printed?
            \vspace{1.5in}
        \item What is the exact number of times that the letter \texttt{y} will be printed?
            \vspace{1.5in}
        \item What is the exact number of times that the letter \texttt{z} will be printed?
        %\item How many times is the letter \texttt{x} printed?
            %\vspace{1.5in}
        %\item How many times is the letter \texttt{y} printed?
            %\vspace{1.5in}
    \end{enumerate}
\end{example}

\newpage
\begin{example}
Answer the questions below based on the following python code:
\begin{python}
for i in range(10):
    print('x')
for i in range(20):
    print('x')

for i in range(10):
    for j in range(20):
        for k in range(30):
            print('y')

print('z')
for i in range(10):
    print('z')
    for j in range(10):
        print('z')
        print('z')
    for j in range(10):
        print('z')
for i in range(10):
    print('z')
\end{python}
    \begin{enumerate}
        \item What is the exact number of times that the letter \texttt{x} will be printed?
            \vspace{1.5in}
        \item What is the exact number of times that the letter \texttt{y} will be printed?
            \vspace{1.5in}
        \item What is the exact number of times that the letter \texttt{z} will be printed?
        %\item How many times is the letter \texttt{x} printed?
            %\vspace{1.5in}
        %\item How many times is the letter \texttt{y} printed?
            %\vspace{1.5in}
        %\item How many times is the letter \texttt{z} printed?
            %\vspace{1.5in}
    \end{enumerate}
\end{example}

\newpage
\begin{example}
Answer the questions below based on the following python code:
\begin{python}
for i in range(n):
    print('x')
for i in range(n*2):
    print('x')

for i in range(n):
    for j in range(n*2):
        for k in range(n*3):
            print('y')

print('z')
for i in range(n):
    print('z')
    for j in range(n):
        print('z')
        print('z')
    for j in range(n):
        print('z')
for i in range(n):
    print('z')
\end{python}
    \begin{enumerate}
        \item What is the exact number of times that the letter \texttt{x} will be printed?
            \vspace{1.5in}
        \item What is the exact number of times that the letter \texttt{y} will be printed?
            \vspace{1.5in}
        \item What is the exact number of times that the letter \texttt{z} will be printed?
        %\item How many times is the letter \texttt{x} printed?
            %\vspace{1.5in}
        %\item How many times is the letter \texttt{y} printed?
            %\vspace{1.5in}
        %\item How many times is the letter \texttt{z} printed?
            %\vspace{1.5in}
    \end{enumerate}
\end{example}

\newpage
\begin{example}
Answer the questions below based on the following python code:
\begin{python}
for i in range(n):
    for j in range(0,i):
        for k in range(0,j):
            print('x')

for i in range(n):
    for j in range(i,n):
        for k in range(j,n):
            print('y')

for i in range(n):
    for j in range(i,n):
        for k in range(i,j):
            print('z')

\end{python}
    \begin{enumerate}
        \item What is the exact number of times that the letter \texttt{x} will be printed?
            \vspace{1.5in}
        \item What is the exact number of times that the letter \texttt{y} will be printed?
            \vspace{1.5in}
        \item What is the exact number of times that the letter \texttt{z} will be printed?
        %\item How many times is the letter \texttt{x} printed?
            %\vspace{1.5in}
        %\item How many times is the letter \texttt{y} printed?
            %\vspace{1.5in}
        %\item How many times is the letter \texttt{z} printed?
            %\vspace{1.5in}
    \end{enumerate}
\end{example}

%%%%%%%%%%%%%%%%%%%%%%%%%%%%%%%%%%%%%%%%%%%%%%%%%%%%%%%%%%%%%%%%%%%%%%%%%%%%%%%%
\newpage
\section{Math: Big-O/$\Theta$/$\Omega$ Notation}

\textbf{Key Ideas:}
\vspace{4in}


\begin{center}
\begin{tikzpicture}
\draw[help lines, color=gray!30, dashed] (-0.5,-0.5) grid (14.9,9.9);
\draw[-,ultra thick] (-0.5,0)--(15,0) node[right]{};
\draw[-,ultra thick] (0,-0.5)--(0,10) node[above]{};
\end{tikzpicture}
\end{center}

\newpage
\begin{defn}
    Let $f,g$ be functions from $\R^+\to\R^+$.
    Then,
    \begin{enumerate}
        \item If $\displaystyle\lim_{x\to\infty} \frac{f(x)}{g(x)} < \infty$, then we say $f = O(g)$.
        \item If $\displaystyle\lim_{x\to\infty} \frac{f(x)}{g(x)} > 0$, then we say $f = \Omega(g)$.
        %\item If $\lim_{x\to\infty} \frac{f(x)}{g(x)} = 0$, then we say $f = O(g)$.
        \item We say that $f = \Theta(g)$ if both $f=O(g)$ and $f=\Omega(g)$.
    \end{enumerate}
    Intuitively, you should think of $O$ as $\le$, $\Omega$ as $\ge$, and $\Theta$ as $=$.
\end{defn}

\begin{example}~
    \begin{enumerate}
        \item 
            $f(x) = x$
            
            $g(x) = x^2$
            \vspace{1in}
        \item
            $f(x) = x^2$
            
            $g(x) = x$
            \vspace{1in}
        \item
            $f(x) = x^2 + 2x + 5$
            
            $g(x) = x$
            \vspace{1in}
        \item
            $f(x) = x^2 + 2x + 5$
            
            $g(x) = x^2$
            \vspace{1in}
        \item
            $f(x) = x^2 + 2x + 5$
            
            $g(x) = x^3$
    \end{enumerate}
\end{example}

\newpage
\begin{example}
    What happens when we double the size of the inputs?
    $f(x) = x$
\end{example}

\newpage
\begin{example}
    You should memorize the relationship between the following functions:
        
        \vspace{0.15in}
        \begin{center}
        $1$\qquad \qquad $\log n$\qquad \qquad  $n$\qquad \qquad  $n\log n$\qquad \qquad  $n^2$\qquad \qquad  $n^3$\qquad \qquad  $2^n$
        \end{center}
        \vspace{0.15in}
\end{example}

\begin{center}
\begin{tikzpicture}
\draw[help lines, color=gray!30, dashed] (-0.5,-0.5) grid (14.9,9.9);
\draw[-,ultra thick] (-0.5,0)--(15,0) node[right]{};
\draw[-,ultra thick] (0,-0.5)--(0,10) node[above]{};
\end{tikzpicture}
\end{center}

\newpage
\begin{example}
    Complete each equation below by adding the symbol $O$ if $f=O(g)$, $\Omega$ if $f=\Omega(g)$, or $\Theta$ if $f=\Theta(g)$.  
    The first row is completed for you as an example.

{\renewcommand{\arraystretch}{4.4}
\begin{tabular}{c c c c c c}
    & f(n) &~\hspace{0.5in}~$ $~\hspace{0.5in}~& g(n) &\\
    \hline
    & $1$ & ~\hspace{0.5in}~$=$~\hspace{0.5in}~  & $O(n)$ &  &\\
    \arrayrulecolor{gray}\hline
    & $3 n\log n$ & ~\hspace{0.5in}~$=$~\hspace{0.5in}~  & $n^2$ &  &\\
    \arrayrulecolor{gray}\hline
    & $1$ & ~\hspace{0.5in}~$=$~\hspace{0.5in}~  & $1/n$ &  &\\
    \arrayrulecolor{gray}\hline
    & $\log_2 n$ & ~\hspace{0.5in}~$=$~\hspace{0.5in}~  & $\log_3 n$ &  &\\
    \arrayrulecolor{gray}\hline
    & $\log n$ & ~\hspace{0.5in}~$=$~\hspace{0.5in}~  & $\frac {1} {\log n}$ &  &\\
    \arrayrulecolor{gray}\hline
    & $5\cdot10^{30}$ & ~\hspace{0.5in}~$=$~\hspace{0.5in}~  & $\log n$ &  &\\
    \arrayrulecolor{gray}\hline
    & $\log n$ & ~\hspace{0.5in}~$=$~\hspace{0.5in}~  & $\log (n^2)$ &  &\\
    \arrayrulecolor{gray}\hline
    & $2^n$ & ~\hspace{0.5in}~$=$~\hspace{0.5in}~  & $3^n$ &  &\\
    \arrayrulecolor{gray}\hline
    & $\frac 1 n$ & ~\hspace{0.5in}~$=$~\hspace{0.5in}~  & $\sqrt{\frac 1 n}$ &  &\\
    \arrayrulecolor{gray}\hline
    & $\log n$ & ~\hspace{0.5in}~$=$~\hspace{0.5in}~  & $(\log n)^2$ &  &\\
    \arrayrulecolor{gray}\hline

    %& $O(1)$ & or & $O(n)$ & or & equal\\
    %& $O(n\log n)$ & or & $O(n^2)$ & or & equal\\
    %& $\Theta(1)$ & or & $\Theta(1/n)$ & or & equal\\
    %& $\Omega(\log_2 n)$ & or & $\Omega(\log_3 n)$ & or & equal\\
    %& $O(n^{42})$ & or & $O(42^n)$ & or & equal\\
    %& $\Theta(5\cdot10^{30})$ & or & $\Theta(\log n)$ & or & equal\\
    %& $\Omega(\log n)$ & or & $\Omega(\log (n^2))$ & or & equal\\
    %& $O(2^n)$ & or & $O(3^n)$ & or & equal\\
    %& $\Theta(n!)$ & or & $\Theta(n^2)$ & or & equal\\
    %& $\Omega(\log n)$ & or & $\Omega((\log n)^2)$ & or & equal\\
\end{tabular}
}
\end{example}

%\newpage
%\begin{problem}
    %Complete each equation below by adding the symbol $O$ if $f=O(g)$, $\Omega$ if $f=\Omega(g)$, or $\Theta$ if $f=\Theta(g)$.  
    %If $f$ cannot be related to $g$ using asymptotic notation, draw a slash through the equals sign.
    %The first row is completed for you as an example.
%
%{\renewcommand{\arraystretch}{4.4}
%\begin{tabular}{c c c c c c}
    %& f(a,b,c) &~\hspace{0.5in}~$ $~\hspace{0.5in}~& g(a,b,c) &\\
    %\hline
    %& $ab$ & ~\hspace{0.5in}~$=$~\hspace{0.5in}~  & $\Omega(b)$ &  &\\
    %\arrayrulecolor{gray}\hline
    %& $a^2b$ & ~\hspace{0.5in}~$=$~\hspace{0.5in}~  & $ab^2$ &  &\\
    %\arrayrulecolor{gray}\hline
    %& $a\log b$ & ~\hspace{0.5in}~$=$~\hspace{0.5in}~  & $a^b$ &  &\\
    %\arrayrulecolor{gray}\hline
    %& $a^2bc^3$ & ~\hspace{0.5in}~$=$~\hspace{0.5in}~  & $a^2b^2c^3$ &  &\\
    %\arrayrulecolor{gray}\hline
    %& $\frac ab$ & ~\hspace{0.5in}~$=$~\hspace{0.5in}~  & $\frac a {b^2}$ &  &\\
    %\arrayrulecolor{gray}\hline
    %& $\frac ab$ & ~\hspace{0.5in}~$=$~\hspace{0.5in}~  & $(\frac a b)^2$ &  &\\
    %\arrayrulecolor{gray}\hline
    %& $a^b$ & ~\hspace{0.5in}~$=$~\hspace{0.5in}~  & $b^a$ &  &\\
    %\arrayrulecolor{gray}\hline
    %& $a^b$ & ~\hspace{0.5in}~$=$~\hspace{0.5in}~  & $(\log a)^c$ &  &\\
    %\arrayrulecolor{gray}\hline
    %& $a^b$ & ~\hspace{0.5in}~$=$~\hspace{0.5in}~  & $(1+c)^a$ &  &\\
    %\arrayrulecolor{gray}\hline
%\end{tabular}
%}
%\end{problem}

\newpage
\begin{example}
    Simplify the following expressions:

\begin{enumerate}
    \item $O\bigg(n^3 + n^2 \bigg)$
    \vspace{1.5in}
    \item $O\bigg(n^3 + 5n^2\log n + \log n \bigg)$
    \vspace{1.5in}
    \item $O\bigg(100000000000 \bigg)$
    \vspace{1.5in}
    \item $O\bigg(\log n + 100000000000 \bigg)$
    \vspace{1.5in}
%\item $O\bigg(ab^2 + 3a^2b + ab + 10b\bigg)$
    %\vspace{1.5in}
%\item $O\bigg(a + b + c + ab + bc + abc\bigg)$
    %\vspace{1.5in}
\item $O\bigg(\frac 1 n + \frac 1 {n^2}\bigg)$
    \vspace{1.5in}
%\item $O\bigg(\frac 1 n + \frac 1 {nm} + \frac 1 m \bigg)$
    %\vspace{1.5in}
%\item $\Omega\bigg((3.45n + n)(\log n^2)\bigg)$
    %\vspace{1.5in}
%\item $\Theta\bigg(n(1 + \log n) + n^{3.2} + \log 2^n\bigg)$
    %\vspace{1.5in}
\end{enumerate}
\end{example}

\vspace{0.25in}
\noindent

%%%%%%%%%%%%%%%%%%%%%%%%%%%%%%%%%%%%%%%%%%%%%%%%%%%%%%%%%%%%%%%%%%%%%%%%%%%%%%%%

\end{document}

