\documentclass[10pt]{article}
\usepackage[utf8]{inputenc}

\usepackage[margin=1in]{geometry}
\usepackage{amsmath}
\usepackage{amssymb}
\usepackage{amsthm}
\usepackage{mathtools}
\usepackage{bm}
\usepackage{tikz}

\usepackage{color}
\usepackage{colortbl}
\definecolor{deepblue}{rgb}{0,0,0.5}
\definecolor{deepred}{rgb}{0.6,0,0}
\definecolor{deepgreen}{rgb}{0,0.5,0}
\definecolor{gray}{rgb}{0.7,0.7,0.7}

\usepackage{hyperref}
\hypersetup{
  colorlinks   = true, %Colours links instead of ugly boxes
  urlcolor     = black, %Colour for external hyperlinks
  linkcolor    = blue, %Colour of internal links
  citecolor    = blue  %Colour of citations
}

%%%%%%%%%%%%%%%%%%%%%%%%%%%%%%%%%%%%%%%%%%%%%%%%%%%%%%%%%%%%%%%%%%%%%%%%%%%%%%%%

\theoremstyle{definition}
\newtheorem{problem}{Problem}
\newtheorem{example}{Example}
\newtheorem{defn}{Definition}
\newcommand{\E}{\mathbb E}
\newcommand{\R}{\mathbb R}
\DeclareMathOperator{\nnz}{nnz}
\DeclareMathOperator{\determinant}{det}
\DeclareMathOperator{\Var}{Var}
\DeclareMathOperator{\rank}{rank}
\DeclareMathOperator*{\argmin}{arg\,min}
\DeclareMathOperator*{\argmax}{arg\,max}

\newcommand{\trans}[1]{{#1}^{T}}
\newcommand{\loss}{\ell}
\newcommand{\w}{\mathbf w}
\newcommand{\x}{\mathbf x}
\newcommand{\y}{\mathbf y}
\newcommand{\lone}[1]{{\lVert {#1} \rVert}_1}
\newcommand{\ltwo}[1]{{\lVert {#1} \rVert}_2}
\newcommand{\lp}[1]{{\lVert {#1} \rVert}_p}
\newcommand{\linf}[1]{{\lVert {#1} \rVert}_\infty}
\newcommand{\lF}[1]{{\lVert {#1} \rVert}_F}

\usepackage{listings}

% Default fixed font does not support bold face
\DeclareFixedFont{\ttb}{T1}{txtt}{bx}{n}{12} % for bold
\DeclareFixedFont{\ttm}{T1}{txtt}{m}{n}{12}  % for normal

% Python style for highlighting
\newcommand\pythonstyle{\lstset{
language=Python,
basicstyle=\ttm,
otherkeywords={self},             % Add keywords here
keywordstyle=\ttb\color{deepblue},
emph={MyClass,__init__},          % Custom highlighting
emphstyle=\ttb\color{deepred},    % Custom highlighting style
stringstyle=\color{deepgreen},
frame=tb,                         % Any extra options here
showstringspaces=false,           % 
stepnumber=1,
numbers=left
}}

\lstnewenvironment{python}[1][]
{
    \pythonstyle
    \lstset{#1}
}
{}

%%%%%%%%%%%%%%%%%%%%%%%%%%%%%%%%%%%%%%%%%%%%%%%%%%%%%%%%%%%%%%%%%%%%%%%%%%%%%%%%

\begin{document}

\begin{center}
    {
\Large
CSCI046 Notes: Stacks
}
\end{center}


%%%%%%%%%%%%%%%%%%%%%%%%%%%%%%%%%%%%%%%%%%%%%%%%%%%%%%%%%%%%%%%%%%%%%%%%%%%%%%%%

\section{Examples}

\newpage
\section{Balancing Parentheses}

\begin{lstlisting}
def balanced_parens(text):
    '''
    Checks whether every ( has a matching ).

    >>> balanced_parens('(()())')
    True
    >>> balanced_parens('(()(()((()))))()(())')
    True
    >>> balanced_parens(')))(((')
    False
    >>> balanced_parens('(()()')
    False
    >>> balanced_parens('(()()))')
    False
    '''
    stack = []
    for i, symbol in enumerate(text):
        if symbol == '(':
            stack.append(symbol)
        else:
            if len(stack) == 0:
                return False
            stack.pop()
    return len(stack) == 0
\end{lstlisting}

\newpage
\begin{lstlisting}
def balanced_parens2(text):
    '''
    Checks whether all of the following types of parentheses are balanced: ([{

    >>> balanced_parens2('(()())')
    True
    >>> balanced_parens2('(()(()((()))))()(())')
    True
    >>> balanced_parens2('([]{})')
    True
    >>> balanced_parens2('([]{[][]{{()}}})')
    True
    >>> balanced_parens2('([}[})')
    False
    >>> balanced_parens2(')))(((')
    False
    >>> balanced_parens2('(()()')
    False
    >>> balanced_parens2('(()()))')
    False
    '''
    stack = []
    for i, symbol in enumerate(text):
        if symbol in '([{' :
            stack.append(symbol)
        else:
            if len(stack) == 0:
                return False
            if (stack[-1] == '(' and symbol == ')') or \
               (stack[-1] == '[' and symbol == ']') or \
               (stack[-1] == '{' and symbol == '}'):
                stack.pop()
            else:
                return False
    return len(stack) == 0
\end{lstlisting}

\end{document}

